\chapter{Methods}

\section{Active Learning Metrics}
The purpose of this paper is investigate how different active learning query metrics can be used to automatically construct learning curricula to improve the generalization performance of deep models. As such we select several active learning approaches, each of which can be used in combination with a curriculum construction methodology (REF SECTION) during training. Testing several methods will also allow us to test the robustness of the results and ascertain whether or not performance differences are consistent across different methods. 

Each active learning method will be used to score the training samples, with the score then being fed as an input into the curriculum construction method to build the training mini-batches throughout the learning phase. As we are not `acquiring' samples, rather we are calculating a score for every training sample, the terminology `acquisition function' would be inappropriate, instead we refer to these score producing functions as \textit{active score functions}.  Each function will map training samples to a real number, which is then passed through a softmax function; this has the effect standardizing the various metrics, as well as allowing us to control the score ratio between different functions using the softmax temperature, as well as producing an output that can interpreted as sampling probabilities (see \ref{Methods_SoftmaxTemperature}). In order to have consistency across the different methods we invert certain scores so that a \textit{higher} score for a sample always corresponding to the sample being estimated to be \textit{easier} for the sample to classify. The classic curriculum learning approach would therefore bias learning towards samples with high scores, while the classic active learning approach would bias learning towards samples with low scores. 

As well as monitoring how the different active score functions affect model performance, we will also investigate how succesful the different methods are at identifying which samples are `hard' or `easy', by visually inspecting the samples which receive very high or low scores by the different functions throughout training. 
\subsection{Average Absolute Distance to Threshold (AADT)}
As laid out in section \ref{Background_ActiveLearning}, a popular active learning method is to examine the proximity to the classification boundary of the model's outputted probabilities (assuming the model outputs probabilities; we will be using deep models with a softmax output layer). The assumption is that samples that the model is uncertain about classifying will produce probabilities close to the classification boundary; indeed as mentioned in section \ref{Background_ActiveLearning} the authors of REF show that prediction variance is inversely proportional to the distance to the boundary. From a curriculum perspective we can estimate a sample's difficulty by the algorithm's uncertainty in predicting the class label, with uncertain samples being seen as hard, and vice versa. 
We therefore calculate the distance to threshold active score function as follows, note that the score is proportional to the \textit{inverse} of the average distance to threshold, in order to ensure that easier samples have a higher score and vice versa:
\begin{equation}
P_{\theta}^{AADT} (x_i) = \frac{exp(\frac{S^{AADT}_{\theta}(x_i)}{\tau})}{\sum_{j}^{N} exp(\frac{S^{DT}_{\theta}(x_j)}{\tau})},
\end{equation}
where
\begin{equation}
S^{AADT}_{\theta}(x_i) = \frac{C}{ \sum_{c}^{C} \left|P_{\theta}(y_c |x_i) - \frac{1}{C}\right|}.
\end{equation}
Where $|.|$ represents the L1 norm/absolute value function.
Here $N$ is the number of training samples, $C$ is the number of output classes and $P_{\theta}(y_c |x_i)$ is the output softmax probability for class $y_c$ of the model parameterised by $\theta$, given input $x_i$. We tried a similar approach using the \textit{square} of the distance to threshold, as opposed to the absolute distance to threshold however results were extremely similar. 

\subsection{Classification Entropy (H)}
Again, as laid in section \ref{Background_ActiveLearning}, a popular uncertainty measure is the entropy of the probabilistic model output. Here, samples which produce outputs with higher entropy represent samples that the model is uncertain about classifying, or, from the curriculum perspective, we see as being `hard' samples. The classification entropy active score function is calculated as follows:

\begin{equation}
P_{\theta}^{H} (x_i) = \frac{exp(\frac{S^{H}_{\theta}(x_i)}{\tau})}{\sum_{j}^{N} exp(\frac{S^{H}_{\theta}(x_j)}{\tau})},
\end{equation}
where
\begin{equation}
S^{H}_{\theta}(x_i) = - \sum_{c}^{C} P_{\theta}(y_c|x)\log P_{\theta}(y_c|x).
\end{equation}

\subsection{BALD}


\subsection{Softmax temperature}\label{Methods_SoftmaxTemperature}
In order to homogenize the outputs of the different active score functions, we pass the the scores through a softmax functions, resulting in an output of softmax probabilities summing to 1. Using the softmax function also allows us to use the softmax temperature in order to control the diversity of the sampling probabilities. A common issue with active learning is that the acquisition functions can end up sampling from an unrepresentative subset of the input space, resulting in significant bias in the training of the model (REF!). Indeed, GIVE EXAMPLE OF MAX RATIO FOR DIST2THRESH

We control this effect by using the softmax temperature to target a preset \textit{maximum probability ratio}, defined as follows:
\begin{equation}
Max Ratio = \frac{\max_{i} P_\theta(x_i)}{\min_{i} P_\theta(x_i)}.
\end{equation}
To do this we begin with a softmax temperature of 1, then calculate the current maximum probability ratio and increment the temperature until the target ratio is achieved. Pseudocode is given below:

PSEUDOCODE FOR SOFTMAX CONTROL

The downside to this method is it could be skewed by outlier probabilities; if there were one sample with an extremely large sampling probability this method would still achieve a target maximum probability ratio, without achieving sufficient diversity in the sampling probabilities. We investigated using winsorization (REF) as an outlier removal technique prior to tuning the temperature parameter, howevever, as the active score functions we use generally did not result in outliers, this did not affect results.  

\section{Curriculum Construction}

\subsection{Biased Sampling}

\subsection{Uniform Sampled Tasks}
Similar to self-paced learning
\subsection{Biased Sampled Tasks}


\section{Datasets}

\subsection{MNIST}

\subsection{Geometric Shapes}

\subsection{CIFAR 10}

\section{Architectures}